\phantomsection
\addcontentsline{toc}{chapter}{Abstract}
\begingroup
\vspace*{\fill}
	\begin{center} 
        \Large \textbf{Abstract}
	\vspace{15pt}
	\end{center}
Interdisciplinary research is often at the core of scientific progress. As an example, artificial neural networks, currently an essential tool in many applications that require learning from data, were originally inspired by insights from biological neural networks. Since its inception as a field, the progress of artificial intelligence has at times converged and at times diverged from the field of neuroscience. While occasional divergence can be fruitful, in this dissertation we will explore some advantageous synergies between machine learning, cognitive science and neuroscience.

In particular, this thesis focuses on vision and images. The human visual system has been widely studied from both behavioural and neuroscientific points of view, as vision is the dominant sense of most people. In turn, machine vision has also been an active area of research, currently dominated by the use of artificial neural networks. Despite their origin and some similarities with biological networks, the recent progress in neural networks for image understanding has shown signs of divergence from neuroscience. One likely cause is the focus on benchmark performance, regardless of \textit{what} the models learn. This work focuses instead on \textit{learning representations} that are more aligned with visual perception and the biological vision. For that purpose, I have studied tools and aspects from cognitive science and computational neuroscience, and attempted to incorporate them into machine learning models of vision.

A central subject of this dissertation is data augmentation, a commonly used technique for training artificial neural networks to \textit{augment} the size of data sets through transformations of the images. Although often overlooked, data augmentation implements transformations that are perceptually plausible, since they correspond to the transformations we see in our visual world---changes in viewpoint or illumination, for instance. Furthermore, neuroscientists have found that the brain invariantly represents objects under these transformations. Throughout this dissertation, I use these insights to analyse data augmentation as a particularly useful inductive bias, a more effective regularisation method for artificial neural networks, and as the framework to analyse and improve the invariance of vision models to perceptually plausible transformations. Overall, this work aims to shed more light on the properties of data augmentation and demonstrate the potential of interdisciplinary research.
\par\vspace{30pt}
\noindent\rule{\linewidth}{0.1mm}\vspace{5pt} % Horizontal Line
\small{The code produced in this thesis is open source and available at \href{https://github.com/alexhernandezgarcia}{\texttt{www.github.com/alexhernandezgarcia}}}
\vspace*{\fill}
\endgroup
