\phantomsection
\addcontentsline{toc}{chapter}{Foreword}
\begingroup
\vspace*{\fill}
	\begin{center} 
        \Large{\textbf{Foreword}}
	\vspace{10pt}
	\end{center}
\small
There are certainly many people I am grateful to for their support and love. I hope I show them and they feel my gratitude and love often enough. Instead of naming them here, I will make sure to thank them personally. I am also grateful to my PhD adviser and collaborators. I acknowledge their specific contributions at the beginning of each chapter. I wish to use this space for putting down in words some thoughts that have been recurrently present throughout my PhD and are especially strong now. 

The main reason why I have been able to write a PhD thesis is \textit{luck}. Specifically, I have been very lucky to be a privileged person. In my opinion, we often overestimate our own effort, skills and hard work as the main factors for professional success. I would not have got here without my privileged position, in the first place. 

There are some obvious comparisons. To name a few: During the first months of my PhD, Syrians had undergone several years of war\footnote{\href{https://en.wikipedia.org/wiki/Syrian_civil_war\#Timeline}{https://en.wikipedia.org/wiki/Syrian\_civil\_war\#Timeline}} and citizens of Aleppo were suffering one of the most cruel sieges ever\footnote{\href{https://en.wikipedia.org/wiki/Battle_of_Aleppo_(2012-2016)}{https://en.wikipedia.org/wiki/Battle\_of\_Aleppo\_(2012-2016)}}. Hundreds of thousands were killed and millions had to flee their land and seek refuge. While I was visiting Palestine after a PhD-related workshop in Jerusalem, 17 Palestinian people were killed by the Israeli army in Gaza during the Land Day\footnote{\href{https://en.wikipedia.org/wiki/Land_Day}{https://en.wikipedia.org/wiki/Land\_Day}} protests. Almost 200 hundred more human beings were killed and thousands were injured by snipers in the weeks thereafter\footnote{\href{https://en.wikipedia.org/wiki/2018-19_Gaza_border_protests}{https://en.wikipedia.org/wiki/2018-19\_Gaza\_border\_protests}}. Also during my PhD, hundreds of thousands of Rohingya\footnote{\href{https://en.wikipedia.org/wiki/Rohingya_refugees_in_Bangladesh}{https://en.wikipedia.org/wiki/Rohingya\_refugees\_in\_Bangladesh}} nationals of Myanmar (formerly Burma) were forced to flee their country due to ethnic and religious persecution. Many people were killed, tortured and thousands of women and girls were raped. An enormous fraction of the population in Africa live below the poverty line, threatened by starvation, conflict, diseases and climate change. Many are forced to cross the continent to seek a safer life in Europe, going through one of the most dangerous migration routes in the world. More than 12,000 human beings have lost their lives in the Mediterranean Sea\footnote{\href{https://data2.unhcr.org/en/situations/mediterranean}{https://data2.unhcr.org/en/situations/mediterranean}} and many more are thought to have died in the Sahara Desert, only since I started my PhD. Around the world, thousands of women are killed by men every year and many live under extremely unsafe conditions\footnote{\href{https://en.wikipedia.org/wiki/Violence_against_women}{https://en.wikipedia.org/wiki/Violence\_against\_women}}. People in these circumstances, regardless of how intelligent and hard-working, can hardly consider the idea of pursuing a PhD. These are only a few well-known, extreme situations around the globe. The list would go much longer, highlighting the privilege of those who are born at the lucky side of the world. I have also been lucky in subtler ways.

First of all, I am a man, and my skin is white. I grew up in Madrid, the capital of a member state of the European Union. This is already privileged with respect to most of the rest of the world, and to most of the rest of my country, Spain. Even though I was not born to a rich family, my parents went to university and became a nurse and a teacher. They always encouraged me and my siblings to study, while giving us the freedom to choose what we wanted. They showed us the pleasure of reading and learning. Not all children, even in the privileged areas of the privileged countries, grow in such positive environments. I have also been lucky that I have not had to go through a separation of my parents. I found supportive friends and had most of the rest of my family close by during my childhood. I worked in the summers to earn some pocket money and certain independence, but I was lucky enough to be able to focus on my studies during the year. Some of my friends could not afford to go to uni. My parents could also afford to cover a year abroad as an Erasmus student. Nothing of this was a merit of mine, but pure \textit{luck}. Without all these positive initial conditions I may not have gained sufficient intellectual and emotional development. Of course I did study and work hard, but many work a lot harder just to survive.

Knowing of my privileged position encourages me to keep the focus and try to make the best out of my time and work. I strongly believe that our everyday actions make a change in the world and that the progress of science will lead us to a fairer planet. This dissertation will make little change, if anything, but the collective effort of many has the potential to make a big impact. May this keep on steering my motivation.
\begin{flushright}
\vspace{5pt}
July 2020
\end{flushright}
\vspace*{\fill}
\endgroup
